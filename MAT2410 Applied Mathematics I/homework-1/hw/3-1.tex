% Problem 4
\subsubsection{4.} 53장의 카드가 들어 있는 주머니에서 비복원추출에 의하여 카드 두 장을 꺼낸다고 하자. 하트의 개수를 $X$, 스페이드의 개수를 $Y$라 한다.

\begin{itemize}
  \item [(1)] $X$와 $Y$의 결합질량함수를 구하라.
  \item [(2)] $X$와 $Y$의 주변질량함수를 구하라.
  \item [(3)] $X$의 평균과 분산을 구하라.
\end{itemize}

\paragraph{Solution.} (1), (2):

\begin{center}
	\begin{tabular}{r|c|c|c|c}
		$X$ / $Y$ & $0$ & $1$ & $2$ & $f_X\left(x\right)$ \\
		\hline
		$0$ & $\dfrac{26}{52}\times \dfrac{25}{51}$ & $2\times \dfrac{13}{52}\times \dfrac{26}{51}$ & $\dfrac{13}{52}\times \dfrac{12}{51}$ & $\dfrac{1482}{2652}$ \\
		\hline
		$1$ & $\dfrac{26}{52}\times \dfrac{13}{51}$ & $2\times \dfrac{13}{52}\times \dfrac{13}{51}$ & $0$ & $\dfrac{1014}{2652}$ \\
		\hline
		$2$ & $\dfrac{13}{52}\times \dfrac{12}{51}$ & $0$ & $0$ & $\dfrac{156}{2652}$ \\
		\hline
		$f_Y\left(y\right)$ & $\dfrac{1482}{2652}$ & $\dfrac{1014}{2652}$ & $\dfrac{156}{2652}$ & $1$ \\
	\end{tabular}
\end{center}

\begin{itemize}
  \item [(3)] $E\left(X\right) = 0.5, E\left(X^2\right) \approx 0.6176, Var\left(X\right) \approx 0.3676$
\end{itemize}

% Problem 5
\subsubsection{5.} 문제 \textbf{4}에서 복원추출을 할 경우, (1)\textasciitilde(3)을 구하라.

\paragraph{Solution.} (1), (2):

\begin{center}
	\begin{tabular}{r|c|c|c|c}
		$X$ / $Y$ & $0$ & $1$ & $2$ & $f_X\left(x\right)$ \\
		\hline
		$0$ & $\dfrac{25}{102}$ & $\dfrac{26}{102}$ & $\dfrac{6}{102}$ & $\dfrac{57}{102}$ \\
		\hline
		$1$ & $\dfrac{26}{102}$ & $\dfrac{13}{102}$ & $0$ & $\dfrac{39}{102}$ \\
		\hline
		$2$ & $\dfrac{6}{102}$ & $0$ & $0$ & $\dfrac{6}{102}$ \\
		\hline
		$f_Y\left(y\right)$ & $\dfrac{57}{102}$ & $\dfrac{39}{102}$ & $\dfrac{6}{102}$ & $1$ \\
	\end{tabular}
\end{center}

\begin{itemize}
  \item [(3)] $E\left(X\right) = 0.5, E\left(X^2\right) \approx 0.6176, Var\left(X\right) \approx 0.3676$
\end{itemize}

% Problem 7
\subsubsection{7.} $X$와 $Y$의 확률질량함수가 다음과 같다. \[f\left(x, y\right) = \dfrac{x+y}{110} \quad x = 1, 2, 3, 4, y = 1, 2, 3, 4, 5\]

\begin{itemize}
  \item [(1)] $X$와 $Y$의 주변질량함수를 구하라.
  \item [(2)] $E\left(X\right)$, $E\left(Y\right)$를 구하라.
  \item [(3)] $P\left(X<Y\right)$
  \item [(4)] $P\left(Y=2X\right)$
  \item [(5)] $P\left(X+Y=5\right)$
  \item [(6)] $P\left(3\leq X+Y\leq 5\right)$
\end{itemize}

\paragraph{Solution.}
\begin{itemize}
  \item [(1)] $\displaystyle f_X\left(x\right) = \sum_{y=1}^5 \dfrac{x+y}{110} = \dfrac{x+3}{22}$

$\displaystyle f_Y\left(y\right) = \sum_{x=1}^4 \dfrac{x+y}{110} = \dfrac{2y+5}{55}$

  \item [(2)] $\displaystyle E\left(X\right) = \sum_{x=1}^4 \sum_{y=1}^5 x\cdot\dfrac{x+y}{110} = \dfrac{30}{11}$

$\displaystyle E\left(Y\right) = \sum_{x=1}^4 \sum_{y=1}^5 y\cdot\dfrac{x+y}{110} = \dfrac{37}{11}$

  \item [(3)] $\displaystyle P\left(X<Y\right) = \sum_{y=2}^5 \sum_{x=1}^{y - 1} \dfrac{x+y}{110} = \dfrac{6}{11}$
  \item [(4)] $\displaystyle P\left(Y=2X\right) = f\left(1, 2\right) + f\left(2, 4\right) = \dfrac{9}{110}$
  \item [(5)] $P\left(X+Y=5\right) = f\left(1, 4\right) + f\left(2, 3\right) + f\left(3, 2\right) + f\left(4, 1\right) = \dfrac{2}{11}$
  \item [(6)] $P\left(3\leq X+Y\leq 5\right) = f\left(1, 2\right) + f\left(2, 1\right) + f\left(1, 3\right) + f\left(2, 2\right) + f\left(3, 1\right) + f\left(1, 4\right) + f\left(2, 3\right) + f\left(3, 2\right) + f\left(4, 1\right) = \dfrac{3}{11}$
\end{itemize}


% Problem 11
\subsubsection{11.} 연속확률변수 $X$와 $Y$의 결합밀도함수가 다음과 같을 때, $X$와 $Y$의 주변밀도함수를 구하라.

\[f\left(x\right) = \left\{
\begin{array}{ll}
	e^{-\left(x + y\right)} & \qquad 0<x<\infty, 0<y<\infty \\
	0 & \qquad\textrm{다른 곳에서}
\end{array}
\right. \]

\paragraph{Solution.} $X$의 주변밀도함수 $f_X$:

\begin{align*}
	f_X\left(x\right) &= \int_0^\infty f\left(x, y\right) \mathop{dy}\\
	&= \int_0^\infty e^{-\left(x + y\right)} \mathop{dy}\\
	&= e^{-x} \int_0^\infty e^{-y} \mathop{dy}\\
	&= e^{-x} \lim_{t\rightarrow\infty} \left. -e^{-y} \right|_0^t\\
	&= e^{-x} \lim_{t\rightarrow\infty} \left(-e^{-t} + e^0\right)\\
	&= e^{-x} \qquad 0<x<\infty
\end{align*}

$Y$의 주변밀도함수 $f_Y$는 같은 방법으로 $0<y<\infty$에서 $f_Y\left(y\right) = e^{-y}$이다.

% Problem 13
\subsubsection{13.} 연속확률변수 $X$와 $Y$의 결합분포함수가 $F\left(x, y\right) = \left(1-e^{-2x}\right)\left(1-e^{-3y}\right)$, $0<x<\infty$, $0<y<\infty$이다.
\begin{itemize}
  \item [(1)] $X$와 $Y$의 결합밀도함수를 구하라.
  \item [(2)] $X$와 $Y$의 주변분포함수와 주변밀도함수를 구하라.
  \item [(3)] 확률 $P\left(1\leq X\leq 2, 0\leq Y\leq 1\right)$을 구하라.
\end{itemize}

\paragraph{Solution.}
\begin{itemize}
  \item [(1)] $X$와 $Y$의 결합밀도함수 $f\left(x, y\right)$:
\begin{align*}
	f\left(x, y\right) &= \dfrac{\mathop{\partial}^2}{\mathop{\partial x}\mathop{\partial y}} F\left(x, y\right) \\
	&= \dfrac{\mathop{\partial}^2}{\mathop{\partial x}\mathop{\partial y}} \left(1-e^{-2x}\right)\left(1-e^{-3y}\right)\\
	&= \left(2e^{-2x}\right)\left(3e^{-3y}\right)\\
	&= 6e^{-2x-3y} \qquad 0<x<\infty, 0<y<\infty
\end{align*}
  \item [(2)] $X$의 주변분포함수 $F_X\left(x\right) = \lim_{y\rightarrow\infty}F\left(x, y\right) = 1-e^{-2x} \qquad 0<x<\infty$
  
  $X$의 주변밀도함수 $f_X\left(x\right) = \dfrac{\mathop{d}}{\mathop{dx}} F_X\left(x\right) = 2e^{-2x} \qquad 0<x<\infty$\\
  
  \item [(3)]
\begin{align*}
	& P\left(1\leq X\leq 2, 0\leq Y\leq 1\right) \\
	&= F\left(2, 1\right) - F\left(2, 0\right) - F\left(1, 1\right) + F\left(1, 0\right)\\
	&= \left(1-e^{-4}\right)\left(1-e^{-3}\right) - \left(1-e^{-4}\right)\left(1-e^{0}\right) - \left(1-e^{-2}\right)\left(1-e^{-3}\right) + \left(1-e^{-2}\right)\left(1-e^{0}\right)\\
	&= \left(1-e^{-4}\right)\left(1-e^{-3}\right) - \left(1-e^{-2}\right)\left(1-e^{-3}\right)\\
	&= e^{-7} - e^{-5} - e^{-4} + e^{-2} \approx 0.1112
\end{align*}
\end{itemize}

% Problem 15
\subsubsection{15.} 연속확률변수 $X$와 $Y$의 결합밀도함수가 $f\left(x, y\right) = \dfrac{3}{16}$, $x^2\leq y\leq 4$, $0\leq x\leq 2$이다.
\begin{itemize}
  \item [(1)] $X$와 $Y$의 주변밀도함수를 구하라.
  \item [(2)] $P\left(1\leq X\leq \sqrt{2}, 1\leq Y\leq 2\right)$
  \item [(3)] $P\left(2X > Y\right)$
\end{itemize}

\paragraph{Solution.}
\begin{itemize}
  \item [(1)] $X$의 주변밀도함수 $f_X$:
\begin{align*}
	f_X\left(x\right) &= \int_{x^2}^4 f\left(x, y\right) \mathop{dy} \\
	&= \int_{x^2}^4 \dfrac{3}{16} \mathop{dy}\\
	&= \dfrac{3}{16}\left(4 - x^2\right) \qquad 0\leq x\leq 2
\end{align*}

$Y$의 주변밀도함수 $f_Y$:
\begin{align*}
	f_Y\left(y\right) &= \int_{\sqrt{y}}^2 f\left(x, y\right) \mathop{dx} \\
	&= \int_{\sqrt{y}}^2 \dfrac{3}{16} \mathop{dx}\\
	&= \dfrac{3}{16}\left(2 - \sqrt{y}\right) \qquad 0\leq y\leq 4
\end{align*}

  \item [(2)]
\begin{align*}
	& P\left(1\leq X\leq \sqrt{2}, 1\leq Y\leq 2\right) \\
	&= \int_1^{\sqrt{2}} \int_{x^2}^2 f\left(x, y\right) \mathop{dy} \mathop{dx}\\
	&= \dfrac{3}{16} \int_1^{\sqrt{2}} \int_{x^2}^2 \mathop{dy} \mathop{dx}\\
	&= \dfrac{3}{16} \int_1^{\sqrt{2}} 2 - x^2 \mathop{dx}\\
	&= \dfrac{3}{16} \left[ 2x - \dfrac{x^3}{3} \right]_1^{\sqrt{2}}\\
	&= \dfrac{4\sqrt{2}-5}{16}
\end{align*}

  \item [(3)]
\begin{align*}
	&P\left(2X > Y\right) \\
	&= \int_0^2 \int_{x^2}^{2x} f\left(x, y\right) \mathop{dy} \mathop{dx}\\
	&= \dfrac{3}{16} \int_0^2 2x - x^2 \mathop{dx}\\
	&= \dfrac{3}{16} \left[ x^2 - \dfrac{x^3}{3} \right]_0^2\\
	&= \dfrac{1}{4}
\end{align*}
\end{itemize}

% Problem 21
\subsubsection{21.} 두 확률변수 $X$와 $Y$의 결합밀도함수가 다음과 같다.
\[f\left(x, y\right) = \left\{
\begin{array}{ll}
	24xy & \qquad 0<y<1-x, 0<x<1 \\
	0 & \qquad\textrm{다른 곳에서}
\end{array}
\right. \]
\begin{itemize}
  \item [(1)] $X$와 $Y$의 주변밀도함수를 구하라.
  \item [(2)] $P\left(X>Y\right)$를 구하라.
\end{itemize}

\paragraph{Solution.} $0<y<1-x \Rightarrow 0<x<1-y$
\begin{itemize}
  \item [(1)] $X$의 주변밀도함수 $f_X$:
\begin{align*}
	f_X\left(x\right) &= \int_0^{1-x} f\left(x, y\right) \mathop{dy} \\
	&= \int_0^{1-x} 24xy \mathop{dy} \\
	&= \left. 12xy^2 \right|_0^{1-x} \\
	&= 12x\left(1-x\right)^2 \qquad 0<x<1
\end{align*}

같은 방법으로 $0<y<1$에서 $Y$의 주변밀도함수 $f_Y\left(y\right) = 12y\left(1-y\right)^2$이다.

  \item [(2)]
\begin{align*}
	& P\left(X>Y\right) \\
	&= \int_0^1 \int_0^{\mathop{\mathrm{min}}\left(x, 1-x\right)} f\left(x, y\right) \mathop{dy} \mathop{dx}\\
	&= \int_0^{\frac{1}{2}} \int_0^{x} f\left(x, y\right) \mathop{dy} \mathop{dx} + \int_{\frac{1}{2}}^1 \int_0^{1 - x} f\left(x, y\right) \mathop{dy} \mathop{dx}\\
	&= 24 \left[ \int_0^{\frac{1}{2}} \int_0^{x} xy \mathop{dy} \mathop{dx} + \int_{\frac{1}{2}}^1 \int_0^{1 - x} xy \mathop{dy} \mathop{dx} \right]\\
	&= 24 \left[ \int_0^{\frac{1}{2}} x \int_0^{x} y \mathop{dy} \mathop{dx} + \int_{\frac{1}{2}}^1 x \int_0^{1 - x} y \mathop{dy} \mathop{dx} \right]\\
	&= 24 \left[ \int_0^{\frac{1}{2}} x\cdot\dfrac{x^2}{2} \mathop{dx} + \int_{\frac{1}{2}}^1 x\cdot\dfrac{\left(1 - x\right)^2}{2} \mathop{dx} \right]\\
	&= 24 \left[ \int_0^{\frac{1}{2}} x\cdot\dfrac{x^2}{2} \mathop{dx} + \int_0^{\frac{1}{2}} \left(1 - x\right)\cdot\dfrac{x^2}{2} \mathop{dx} \right]\\
	&= 24 \int_0^{\frac{1}{2}} \dfrac{x^2}{2} \mathop{dx}\\
	&= 4 \left.x^3\right|_0^{\frac{1}{2}}\\
	&= \dfrac{1}{2}
\end{align*}
\end{itemize}

% Problem 26
\subsubsection{26.} 두 확률변수 $X$와 $Y$의 결합밀도함수가 $f\left(x, y\right) = 3e^{-x-3y}$, $x>0$, $y>0$이다.
\begin{itemize}
  \item [(1)] 결합분포함수 $F\left(x, y\right)$를 구하라.
  \item [(2)] $X$와 $Y$의 주변밀도함수를 구하라.
  \item [(3)] $P\left(x<y\right)$를 구하라.
\end{itemize}

\paragraph{Solution.} 
\begin{itemize}
  \item [(1)] 결합분포함수 $F\left(x, y\right)$:
\begin{align*}
	& F\left(x, y\right) \\
	&= \int_0^x \int_0^y f\left(u, v\right) \mathop{dv} \mathop{du} \\
	&= 3 \int_0^x \int_0^y e^{-u-3v} \mathop{dv} \mathop{du} \\
	&= 3 \int_0^x e^{-u} \int_0^y e^{-3v} \mathop{dv} \mathop{du} \\
	&= -\left(e^{-3y} - 1\right) \int_0^x e^{-u} \mathop{du} \\
	&= \left(e^{-x} - 1\right)\left(e^{-3y} - 1\right) \qquad x>0, y>0
\end{align*}

  \item [(2)] $X$의 주변밀도함수 $f_X$:
\begin{align*}
	f_X\left(x\right) &= \int_0^\infty f\left(x, y\right) \mathop{dy} \\
	&= \int_0^\infty 3e^{-x-3y} \mathop{dy} \\
	&= 3e^{-x} \int_0^\infty e^{-3y} \mathop{dy} \\
	&= 3e^{-x} \lim_{t\rightarrow\infty} \int_0^t e^{-3y} \mathop{dy} \\
	&= -e^{-x} \lim_{t\rightarrow\infty} \left(e^{-3t} - e^0\right) \\
	&= e^{-x}  \qquad 0<x<\infty
\end{align*}

$Y$의 주변밀도함수 $f_Y$:
\begin{align*}
	f_Y\left(y\right) &= \int_0^\infty f\left(x, y\right) \mathop{dx} \\
	&= \int_0^\infty 3e^{-x-3y} \mathop{dx} \\
	&= 3e^{-3y} \int_0^\infty e^{-x} \mathop{dx} \\
	&= 3e^{-3y} \lim_{t\rightarrow\infty} \int_0^t e^{-x} \mathop{dx} \\
	&= -3e^{-3y} \lim_{t\rightarrow\infty} \left(e^{-t} - e^0\right) \\
	&= 3e^{-3y}  \qquad 0<y<\infty
\end{align*}

  \item [(3)]
\begin{align*}
	& P\left(x<y\right) \\
	&= \int_0^\infty \int_x^\infty f\left(x, y\right) \mathop{dy} \mathop{dx} \\
	&= \int_0^\infty \int_x^\infty 3e^{-x-3y} \mathop{dy} \mathop{dx} \\
	&= \int_0^\infty 3e^{-x} \int_x^\infty e^{-3y} \mathop{dy} \mathop{dx} \\
	&= \int_0^\infty 3e^{-x} \left(\dfrac{1}{3} e^{-3x}\right) \mathop{dx} \\
	&= \int_0^\infty e^{-4x} \mathop{dx} \\
	&= \dfrac{1}{4}
\end{align*}
\end{itemize}
 
% Problem 31 
\subsubsection{31.} 보험회사는 대단히 많은 운전자를 가입자로 가지고 있다. 자동차 충돌에 의한 보험회사의 손실을 확률변수 $X$라 하고, 책임보험에 의한 손실을 $Y$라고 하자. 두 확률변수의 결합밀도함수가 다음과 같을 때, 두 손실의 총액이 적어도 1 이상일 확률을 구하라.

\[f\left(x, y\right) = \left\{
\begin{array}{ll}
	\dfrac{2x+2-y}{4} & \qquad 0<x<1, 0<y<2 \\
	0 & \qquad\textrm{다른 곳에서}
\end{array}
\right. \]

\paragraph{Solution.} 구하고자 하는 값은 $P\left(X+Y\geq 1\right)$이다. $P\left(X+Y\geq 1\right) = 1 - P\left(X+Y<1\right)$이므로, 구하고자 하는 확률은 $P\left(X+Y<1\right)$을 구해 해결할 수 있다. $x+y<1$이면 $y<1-x$이다.

\begin{align*}
	& P\left(X+Y<1\right) \\
	&= \int_0^1 \int_0^{1-x} f\left(x, y\right) \mathop{dy} \mathop{dx} \\
	&= \int_0^1 \int_0^{1-x} \dfrac{2x+2-y}{4} \mathop{dy} \mathop{dx} \\
	&= \int_0^1 \left[ \dfrac{4xy+4y-y^2}{8} \right]_{y=0}^{y=1-x} \mathop{dx} \\
	&= \int_0^1 \dfrac{-5x^2+2x+3}{8} \mathop{dx} \\
	&= \dfrac{1}{8} \left[ -\dfrac{5}{3}x^3+x^2+3x \right]_0^1 \\
	&= \dfrac{7}{24}
\end{align*}

따라서 $P\left(X+Y\geq 1\right) = 1 - P\left(X+Y<1\right) = \dfrac{17}{24}$이다.