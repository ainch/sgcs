
\documentclass[runningheads]{llncs}
\usepackage[paperheight=295mm,paperwidth=210mm]{geometry}
\usepackage{graphicx}
\usepackage{kotex}
\usepackage{amsmath}
\usepackage{amssymb}
\usepackage[dvipsnames]{xcolor}
\usepackage{fancyvrb}
\usepackage{listings}
\usepackage{indentfirst}
\usepackage{tabularx}
\usepackage{underscore}
\usepackage{multicol}
\usepackage{tikz}
\usepackage[square,sort,comma,super]{natbib}
\usepackage{inconsolata} % Inconsolata
\usepackage{mathptmx} % Times New Roman
\usepackage[cache=false]{minted}
\graphicspath{ {./images/} }
\lstset{basicstyle=\footnotesize\ttfamily,breaklines=true}
\renewcommand{\bibname}{참고문헌}
\setlength{\parindent}{1em}
\setlength{\parskip}{1em}
\linespread{1.2}
{\renewcommand{\arraystretch}{1.5}%
\setlength{\tabcolsep}{0.5em}%
\newenvironment{Figure}
  {\par\medskip\noindent\minipage{\linewidth}}
  {\endminipage\par\medskip}
	
\begin{document}

\title{STS2006 (Analytic Geometry and Calculus II) \newline Quiz 2 Solutions}
\author{Suhyun Park (20181634)}
\institute{Department of Computer Science and Engineering, Sogang University}
\maketitle

\subsubsection{1. (5 pts)} Find the approximation of $\sqrt{\left(3.02\right)^2+\left(1.97\right)^2+\left(5.99\right)^2}$ by using the linear approximation method.

\paragraph{Solution.} Let \[w = f\left(x, y, z\right) = \sqrt{x^2+y^2+z^2}\]
then
\begin{align*}
	&\sqrt{\left(3.02\right)^2+\left(1.97\right)^2+\left(5.99\right)^2}\\
	\approx & f\left(3, 2, 6\right)
		+\dfrac{\partial w}{\partial x}\mathop{dx}
		+\dfrac{\partial w}{\partial y}\mathop{dy}
		+\dfrac{\partial w}{\partial z}\mathop{dz}\\
	=&\sqrt{3^2+2^2+6^2}
		+\left(\dfrac{2\times3}{2\sqrt{3^2+2^2+6^2}}\right)\mathop{dx}
		+\left(\dfrac{2\times2}{2\sqrt{3^2+2^2+6^2}}\right)\mathop{dy}
		+\left(\dfrac{2\times6}{2\sqrt{3^2+2^2+6^2}}\right)\mathop{dz}\\
	=&7+\dfrac{3}{7}\mathop{dx}+\dfrac{2}{7}\mathop{dy}+\dfrac{6}{7}\mathop{dz}\\
	=&7+\dfrac{3}{7}\left(0.02\right)+\dfrac{2}{7}\left(-0.03\right)+\dfrac{6}{7}\left(-0.01\right)\\
	=&7-\dfrac{6}{700}
\end{align*}
\par

\subsubsection{2. (5 pts)} Use the chain rule to find $\dfrac{\partial w}{\partial r}$, $\dfrac{\partial w}{\partial \theta}$ where $w=xy+yz+zx$, $x=r\cos\theta$, $y=r\sin\theta$, $z=r\theta$.

\paragraph{Solution.}
\begin{align*}
	\dfrac{\partial w}{\partial r}
	&= \dfrac{\partial w}{\partial x}\dfrac{\partial x}{\partial r}
		 + \dfrac{\partial w}{\partial y}\dfrac{\partial y}{\partial r}
		 + \dfrac{\partial w}{\partial z}\dfrac{\partial z}{\partial r}\\
	&= \left(y+z\right)\cos\theta+\left(x+z\right)\sin\theta+\left(x+y\right)\theta
\end{align*}

\begin{align*}
	\dfrac{\partial w}{\partial \theta}
	&= \dfrac{\partial w}{\partial x}\dfrac{\partial x}{\partial \theta}
		 + \dfrac{\partial w}{\partial y}\dfrac{\partial y}{\partial \theta}
		 + \dfrac{\partial w}{\partial z}\dfrac{\partial z}{\partial \theta}\\
	&= \left(y+z\right)\left(-r\sin\theta\right)+\left(x+z\right)\left(r\cos\theta\right)+\left(x+y\right)r
\end{align*}

\end{document}
