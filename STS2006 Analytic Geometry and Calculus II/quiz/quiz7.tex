
\documentclass[runningheads]{llncs}
\usepackage[paperheight=295mm,paperwidth=210mm]{geometry}
\usepackage{graphicx}
\usepackage{kotex}
\usepackage{amsmath}
\usepackage{amssymb}
\usepackage[dvipsnames]{xcolor}
\usepackage{fancyvrb}
\usepackage{listings}
\usepackage{indentfirst}
\usepackage{tabularx}
\usepackage{underscore}
\usepackage{multicol}
\usepackage{tikz}
\usepackage[square,sort,comma,super]{natbib}
\usepackage{inconsolata} % Inconsolata
\usepackage{mathptmx} % Times New Roman
\usepackage[cache=false]{minted}
\graphicspath{ {./images/} }
\lstset{basicstyle=\footnotesize\ttfamily,breaklines=true}
\renewcommand{\bibname}{참고문헌}
\setlength{\parindent}{1em}
\setlength{\parskip}{1em}
\linespread{1.2}
{\renewcommand{\arraystretch}{1.5}%
\setlength{\tabcolsep}{0.5em}%
\newenvironment{Figure}
  {\par\medskip\noindent\minipage{\linewidth}}
  {\endminipage\par\medskip}
	
\begin{document}

\title{STS2006 (Analytic Geometry and Calculus II) \newline Quiz 7 Solutions}
\author{Suhyun Park (20181634)}
\institute{Department of Computer Science and Engineering, Sogang University}
\maketitle

\subsubsection{1. (5 pts)} Evaluate the line integral \[\int_c \left(y+e^{x^2+2x+1}\right)\mathop{dx}+\left(ex+\cos y^2+\sin\left(y+1\right)\right)\mathop{dy}\] where $C$ is positively oriented boundary curve of a region $D$ that has area of 10. \textit{(Hint: Use Green's theorem.)}

\paragraph{Solution.}
\begin{align*}
	& \int_c \left(y+e^{x^2+2x+1}\right)\mathop{dx}+\left(ex+\cos y^2+\sin\left(y+1\right)\right)\mathop{dy}\\
	=& \iint_D \dfrac{\partial}{\partial x}\left(ex+\cos y^2+\sin\left(y+1\right)\right) 
		- \dfrac{\partial}{\partial y}\left(y+e^{x^2+2x+1}\right)\mathop{dA}\\
	=& \iint_D e-1\mathop{dA} = 10\left(e-1\right)
\end{align*}
\par

\subsubsection{2. (5 pts)} Find a parametric representation for the part of the cylinder $y^2+z^2=16$ that lies between the planes $x=0$ and $x=5$.

\paragraph{Solution.} \[
	\begin{cases}
		x=x \\
		y=4\cos\theta \\
		z=4\sin\theta
	\end{cases}\quad\text{where}\quad 0<x<5, 0\leq\theta\leq2\pi
\]

\end{document}
